\documentclass[a5paper, 10pt]{article}

\usepackage[utf8]{inputenc}
\usepackage[brazilian]{babel}

% The following packages can be found on http:\\www.ctan.org
\usepackage{graphics} % for pdf, bitmapped graphics files
\usepackage{epsfig} % for postscript graphics files
\usepackage{mathptmx} % assumes new font selection scheme installed
\usepackage{times} % assumes new font selection scheme installed
\usepackage{amsmath} % assumes amsmath package installed
\usepackage{amssymb}  % assumes amsmath package installed

\title{\LARGE \bf
Formalização e Propriedades do Algoritmo \textit{Radix Sort}
}

\author{Diogo, Eduardo, Gabriel e Juliana Hosoume}

\begin{document}
\maketitle

\begin{abstract}

Neste trabalho mostraremos diversas utilizações do princípio de indução em Ciência da Computação.

\end{abstract}

\section{Introdução e Contextualização do Problema}

\section{Explicação das Soluções}

\section{Especificação do Problema e Explicação do Método de Solução}

\section{Explicação da Formalização}

\section{Conclusões}

\end{document}
